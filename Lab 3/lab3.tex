\documentclass[b5paper,11pt]{article}
\usepackage[utf8]{inputenc}
\usepackage[T1]{fontenc}
\usepackage[polish]{babel} % Język polski
\usepackage{geometry} % Pakiet do zarządzania geometrią strony
\usepackage{soul} % Pakiet do podkreśleń
\usepackage{ulem} % Pakiet do przekreśleń
\usepackage{setspace} % Pakiet do interlinii
\usepackage{amsmath} % Pakiet do równań
\usepackage{graphicx} % Pakiet do grafiki
\usepackage{xcolor} % Pakiet do kolorów
\usepackage{hyperref} % Pakiet do linków
\usepackage{cite} % Pakiet do zarządzania cytowaniami



\geometry{left=3cm, right=2cm, top=2cm, bottom=2cm}

\begin{document}
\begin{spacing}{1.5} % Interlinia 1.5

    \section{Zadanie 1 - formatowanie tekstu}

    Możemy zastosować różne style formatowania tekstu, na przykład:

    \textbf{Pogrubienie} - Tekst jest pogrubiony.

    \uline{Podkreślenie} - Tekst jest podkreślony używając pakietu \texttt{ulem}.

    \textit{Kursywa} - Tekst jest kursywą.

    \textbf{\uline{Pogrubienie i podkreślenie}} - Tekst jest jednocześnie pogrubiony i podkreślony.

    \texttt{Typewriter} - Tekst jest w stylu czcionki typewriter.

    \textbf{\textit{Pogrubienie i kursywa}} - Tekst jest jednocześnie pogrubiony i kursywą.

    Możemy także zmieniać rozmiary czcionki, na przykład: {\tiny tekst bardzo mały}, {\scriptsize tekst mniejszy}, {\footnotesize tekst jeszcze mniejszy}, {\small tekst mały}, {\large tekst duży}, {\Large tekst większy}, {\LARGE tekst jeszcze większy}, {\huge tekst bardzo duży}, {\Huge tekst ogromny}.


    \newpage
    \section{Zadanie 2 - równania matematyczne}

    Przykładowe równanie:

    \[
        \sin^2 x + \cos^2 x = 1
    \]

    Przykładowe zależności:

    \begin{equation}
        \gamma^2 + \theta^2 = \omega^2 \label{eq:example1}
    \end{equation}

    Poniżej przedstawiono równania Maxwella:

    \begin{equation}
        \vec{\nabla} \cdot \vec{E} = \frac{\rho}{\epsilon_0} \label{eq:maxwell1}
    \end{equation}

    \begin{equation}
        \vec{\nabla} \cdot \vec{B} = 0 \label{eq:maxwell2}
    \end{equation}

    \begin{equation}
        \vec{\nabla} \times \vec{E} = -\frac{\partial \vec{B}}{\partial t} \label{eq:maxwell3}
    \end{equation}

    \begin{equation}
        \vec{\nabla} \times \vec{B} = \mu_0 \left(\epsilon_0 \frac{\partial \vec{E}}{\partial t} + \vec{J}\right) \label{eq:maxwell4}
    \end{equation}

    Przykładowe odwołania do tych równań:

    Równanie \eqref{eq:example1} wykorzystane zostało do obliczenia pola magnetycznego. Równania \eqref{eq:maxwell1}, \eqref{eq:maxwell2}, \eqref{eq:maxwell3} oraz \eqref{eq:maxwell4} opisują podstawowe prawa elektromagnetyki.

    \newpage
    \section{Zadanie 3 - równanie macierzowe}

    \begin{equation}
        \begin{pmatrix}
            a_{11} & a_{12} & \ldots & a_{1n} \\
            a_{21} & a_{22} & \ldots & a_{2n} \\
            \vdots & \vdots & \ddots & \vdots \\
            a_{n1} & a_{n2} & \ldots & a_{nn} \\
        \end{pmatrix}
        \begin{bmatrix}
            b_{1}  \\
            b_{2}  \\
            \vdots \\
            b_{n}  \\
        \end{bmatrix}
        =
        \begin{matrix}
            c_{1}  \\
            c_{2}  \\
            \vdots \\
            c_{n}  \\
        \end{matrix}
    \end{equation}

    \newpage
    \section{Zadanie 4 - dodawanie tabel oraz rysunków}

    Poniżej przedstawiono tabelę, która zawiera 4 kolumny i 3 wiersze wypełnione przykładowymi danymi:

    \begin{table}[h]
        \centering
        \caption{Obliczenia dla funkcji $f(x) = 4 \cdot x$}
        \begin{tabular}{|c|c|c|c|}
            \hline
            $x$    & 1 & 2 & 3  \\
            \hline
            $f(x)$ & 4 & 8 & 12 \\
            \hline
        \end{tabular}
    \end{table}

    W tabeli 2 zestawiono...

    Odwołanie do rysunku poniżej (rys. 1). Aby zachować skalę, szerokość rysunku zostanie ustawiona na połowę globalnej szerokości obszaru tekstowego (\texttt{textwidth}), a następnie rysunek zostanie obrócony o kąt 45°:

    \begin{figure}[h]
        \centering
        \includegraphics[width=0.5\textwidth, angle=45]{images/logotyp-politechnika-opolska-01.jpg} % Zastąp 'example-image.png' nazwą swojego pliku z rysunkiem
        \caption{Logo uczelniane.}
    \end{figure}

    \newpage
    \section{Zadanie 5 - listy numerowane i nienumerowane}

    Przykład listy numerowanej:
    \begin{enumerate}
        \item pierwsza
        \item druga
        \item trzecia
    \end{enumerate}

    Przykład listy nienumerowanej:
    \begin{itemize}
        \item pierwsza
        \item druga
        \item trzecia
    \end{itemize}

    \newpage
    \section{Zadanie 6 - formatowanie dokumentu}

    To jest przykład tekstu z przypisem\footnote{To jest treść przypisu.}.

    Odwiedź naszą stronę internetową: \href{https://www.example.com}{www.example.com}

    Skontaktuj się z nami pod adresem: \href{mailto:kontakt@example.com}{kontakt@example.com}

    Przykład~tekstu~z~twardą~spacją.

    To jest \textbf{pogrubiony tekst}.

    To jest \textcolor{red}{czerwony tekst}.
    \begin{center}
        To jest tekst wyśrodkowany.
    \end{center}

    \newpage
    \section{Zadanie 7 - dodanie literatury}

    Cytowanie 1 \cite{Kowol2009}.

    Cytowanie 2 \cite{Mynarek2011}.

    \bibliographystyle{plain} % Wybierz styl bibliografii
    \bibliography{literatura} % Wskaż nazwę pliku .bib

    \newpage
    \section{Zadanie 8 - dodanie spisu treści}

    % Spis treści
    \tableofcontents

    % Lista tabel
    \listoftables

    % Lista rysunków
    \listoffigures

\end{spacing}
\end{document}
